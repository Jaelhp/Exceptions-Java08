\documentclass[11pt]{beamer}
\usepackage{listings} % Include the listings-package
\usepackage[T1]{fontenc}
\usepackage[utf8]{inputenc}
\usepackage[english]{babel}
\usepackage{amsmath}
\usepackage{amssymb, amsfonts, latexsym, cancel}
\usepackage{float}
\usepackage{graphicx}
\usepackage{epstopdf}
\usepackage{subfigure}
\usepackage{hyperref}
%\usepackage{authblk}
\usepackage{blindtext}
\usepackage{booktabs} % Allows the use of \toprule, 
\usepackage{filecontents}
\usepackage{courier} %% Sets font for listing as Courier.
\usepackage{listings}
%\usepackage{listings, xcolor}
\lstset{
tabsize = 2, %% set tab space width
showstringspaces = false, %% prevent space marking in strings, string is defined as the text that is generally printed directly to the console
numbers = left, %% display line numbers on the left
commentstyle = \color{green}, %% set comment color
keywordstyle = \color{blue}, %% set keyword color
stringstyle = \color{red}, %% set string color
rulecolor = \color{black}, %% set frame color to avoid being affected by text color
basicstyle = \small \ttfamily , %% set listing font and size
breaklines = true, %% enable line breaking
numberstyle = \tiny,
}
\usepackage{caption}
\DeclareCaptionFont{white}{\color{white}}
\DeclareCaptionFormat{listing}{\colorbox{gray}{\parbox{\textwidth}{#1#2#3}}}
\captionsetup[lstlisting]{format=listing,labelfont=white,textfont=white}
\definecolor{urlColor}{rgb}{0.06, 0.3, 0.57}
\definecolor{linkColor}{rgb}{0.57, 0.0, 0.04}
\definecolor{fileColor}{rgb}{0.0, 0.26, 0.26}
\hypersetup{
    colorlinks=true,
    linkcolor=linkColor,
    filecolor=fileColor,      
    urlcolor=urlColor,
}
\urlstyle{same}
\setbeamercovered{transparent}
%\usetheme{Boadilla}
\usetheme{CambridgeUS}
%\usetheme{Berkeley}
%\usetheme{Warsaw}
%\usetheme{Madrid}
\title[Exceptions]{\bf\Huge Exceptions in Java }
\subtitle{Fundamentals to programming I}

\author[Jaelhp]
{
	Jael Emerson Huarca Pallani \inst{1}
}
\institute[UNSA]
{
\inst{1}% 
System Engineering School\\
System Engineering and Informatic Department\\
Production and Services Faculty\\
San Agustin National University of Arequipa
}
\date[2020-08-01]{\scriptsize{2020-08-01}}
%\logo{\includegraphics[width=3.0cm]{Img/logo_unsa.jpg} }
\titlegraphic{\includegraphics[width=3.0cm]{Img/logo_unsa.jpg} }

\begin{document}
\begin{frame}
\titlepage
\end{frame}

\begin{frame}
\frametitle{Content}
\tableofcontents
\end{frame}

\section{Exceptions - Definition}
\begin{frame}
\frametitle{Exception - Definition}
\begin{itemize}
\item 

An exception is an abnormal situation that can occur when we run a certain program.
Exceptions are a way of trying to ensure that if a source code does not run as intended initially, the programmer is able to control that situation and say how the program should respond.

\end{itemize}
\end{frame}
 
\section{Hierarchy of exceptions}
\begin{frame}
\frametitle{Hierarchy of exceptions}
\begin{center}
{\includegraphics[width=8.0cm]{Img/Exception-in-java.png}  }
\end{center}
\end{frame}

\section{Examples - Basics}
\begin{frame}
\frametitle{Examples - Basics}
\begin{center}
{\includegraphics[width=8.0cm]{Img/Examplebasic.PNG}   }
\end{center}
\end{frame}

\section{Examples - Basics1}
\begin{frame}
\frametitle{Examples - Basics1}
\begin{center}
{\includegraphics[width=8.0cm]{Img/Examplebasic1.png}  }
\end{center}
\end{frame}


\section{Example - Advanced}
\begin{frame}
\frametitle{Example - Advanced}
\begin{center}
{\includegraphics[width=8.0cm]{Img/ExampleAdvanced.PNG}   }
\end{center}
\end{frame}




\section{References}
%References frame
\begin{frame}
\frametitle{References - Web pages}
\begin{itemize}

\item \url{https://www.oracle.com/java/technologies/javase/javase-jdk8-downloads.html}
\item \url{https://www.eclipse.org/downloads/packages/release/2020-06/r/eclipse-ide-enterprise-java-developers}
\item \url{https://www.youtube.com/watch?v=jYnHBbqHE3A}
\item \url{https://elvex.ugr.es/decsai/java/}
\item \url{https://www.youtube.com/watch?v=mCmu7Ps55Dc}
\item \url{https://www.unirioja.es/cu/jearansa/0910/archivos/EIPR_Tema05.pdf}
\end{itemize}
\end{frame}


\begin{frame}
\begin{center}
Thanks!

\end{center}
\end{frame}

\end{document}